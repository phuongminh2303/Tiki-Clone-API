\documentclass{beamer}

\mode<presentation>
{
  \usetheme{Madrid}      
  \usecolortheme{beaver} 
  \usefonttheme{serif} 
  \setbeamertemplate{navigation symbols}{}
  \setbeamertemplate{caption}[numbered]
} 

\usepackage[english]{babel}
\usepackage{graphicx}
\usepackage{hyperref}
\usepackage[utf8x]{inputenc}
\usepackage{xcolor}
\usepackage{listings}
\hypersetup{
    colorlinks=true,
    linkcolor=black,
    filecolor=magenta,      
    urlcolor=blue,
}
\setbeamertemplate{itemize items}[circle]
\lstset
{
    language=[LaTeX]TeX,
    breaklines=true,
    basicstyle=\tt\scriptsize,
    keywordstyle=\color{blue},
    identifierstyle=\color{magenta},
}


\title[Tiki clone]{Web Application Development}
\author{Group 2}
\subtitle{Project presentation: Tiki clone}
\institute{ICT dept/USTH}
\date{March 28, 2020}

\AtBeginSection[]
{
  \begin{frame}<beamer>
    \frametitle{Outline}
    \tableofcontents[currentsection,currentsubsection]
  \end{frame}
}

\begin{document}

\begin{frame}
  \titlepage
\end{frame}

% Uncomment these lines for an automatically generated outline.
\begin{frame}{Outline}
  \tableofcontents
\end{frame}

\section{Introduction}

\begin{frame}{Motivation}
	Vietnames Ecommerce Statistics here
\end{frame}

\begin{frame}{Why Tiki-clone ?}
	\begin{itemize}
  		\item Tiki.vn is a popular and favorite commercial website in Vietnam.
  		\item Tiki's system architecture is complicated.
  		\begin{itemize}
			\item Most of the time, they can handle a huge amount of requests in their server.
			\item They can scale both horizontally and vertically very fast
			\item They use modern technologies/frameworks: ReactJS for example.
		\end{itemize}
  		\item Tiki only supports Vietnamese. But Tiki-clone version will support English.
	\end{itemize}
\end{frame}

\section{System Architecture}

\begin{frame}{Use case design and analysis}
\begin{figure}[htp]
    \centering
    \includegraphics[scale=0.3]{images/usecase.png}
    \caption{Use case diagram}
    \label{fig:usecase}
\end{figure}
\end{frame}

\begin{frame}{System architecture design}
\begin{figure}[htp]
    \centering
    \includegraphics[scale=0.3]{images/usecase.png}
    \caption{System architecture}
    \label{fig:sysarch}
\end{figure}
\end{frame}

\begin{frame}{Role-based access control design}
\begin{figure}[htp]
    \centering
    \includegraphics[scale=0.3]{images/usecase.png}
    \caption{Role-based access control}
    \label{fig:sysarch}
\end{figure}
\end{frame}

\begin{frame}{Database design}
\begin{figure}[htp]
    \centering
    \includegraphics[scale=0.3]{images/usecase.png}
    \caption{Role-based access control}
    \label{fig:sysarch}
\end{figure}
\end{frame}

\begin{frame}{Tools and frameworks}
    \begin{columns}
    \linespread{1.0}
    \begin{column}{5cm}
        \textbf{Front-end:}
        \begin{itemize}
            \item ReactJS with Redux
            \item Material UI
        \end{itemize}
        \textbf{Back-end: }
        \begin{itemize}
            \item NodeJS with Express
            \item MongoDB Atlas
            \item Redis
        \end{itemize}
        \textbf{Server: }
        \begin{itemize}
            \item Google Cloud CE
            \item Nginx
            \item Docker (A little bit)
        \end{itemize}
    \end{column}
    \begin{column}{5cm}

    \end{column}
    \end{columns}
\end{frame}

\section{Code Implementation}
	
\begin{frame}{Code Implementation}
Sample ReactJS and Redux "Login" implementation
\end{frame}

\begin{frame}{Code Implementation}
Mongoose schema implementation
Sample get all product route + controller
\end{frame}

\begin{frame}[fragile]{Server-side caching and Load balancing}
\begin{itemize}
    \item Sample redis implemenation + keys + response time result
    \item Nginx configuration file
\end{itemize}
\end{frame}

\begin{frame}[fragile]{Security}
\begin{itemize}
	\item	Encrypt user's passwords and tokens
	\item	Protect against NoSQL injections
	\item	Use helmet package to prevent cross site scripting (XSS)
	\item	Add rate limit (100 requests/ 10 mins) (maybe change later)
	\item	Prevent HTTP param polution
	\item	Public the API with CORS
	\end{itemize}
\end{frame}

\section{Results and Discussion}

\begin{frame}{Results}
	\begin{itemize}
    	\item Tiki-clone web-based application
    	\begin{itemize}
    	    \item Single web page application
    	    \item Responsive with all devices!
    	\end{itemize}
    	\pause
    	\item Well constructed JSON-based API and documentations
    	\begin{itemize}
    	    \item Scalable
    	    \item Quite security (at the moment)
    	\end{itemize}
    	\pause
        \item Web server architecture that can handle multiple requests
        \item Finish all use cases in the design part
	\end{itemize}
\end{frame}

\begin{frame}[fragile]
\frametitle{Result images here}
    Images here
\end{frame} 
    
\begin{frame}[fragile]
\frametitle{Difficulties}
    \begin{itemize}
    	\item Interrupted class timetable and schedule due to the disease
    	\item Medium/low technical skills with tools and frameworks
        \item Free tier Google Cloud server is not strong enough
        \item Bad team management from the leader
	\end{itemize}
\end{frame}    

\begin{frame}[fragile]
\frametitle{Future improvements}
    \begin{itemize}
    	\item Refactor code using SOLID/DRY technique
    	\item Improve system architecture: 
    	    \begin{itemize}
    	        \item Add more WS and LB
    	        \item Transfer to microservices
    	        \item Containerize the system
    	    \end{itemize}
        \item Add domain name and encrypt SSL
        \item Resolve our issues that are noted in Github repos
	\end{itemize}
\end{frame}

\section{Demo and resources}
	
\begin{frame}{Demo and resources}
	\begin{itemize}
	    \item Tiki-clone on Heroku: \url{heroku app link}
	    \item Front-end repository: \url{https://github.com/phuongminh2303/Tiki-Clone}
	    \item Back-end repository: \url{https://github.com/phuongminh2303/Tiki-Clone-API}
	    \item API documentation: \url{https://documenter.getpostman.com/view/9718206/SzKN2hcs}
	\end{itemize}
\end{frame}

\end{document}
